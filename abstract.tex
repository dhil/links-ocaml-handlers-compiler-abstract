\documentclass[preprint,numbers]{sigplanconf}
% The following \documentclass options may be useful:

% preprint      Remove this option only once the paper is in final form.
% 10pt          To set in 10-point type instead of 9-point.
% 11pt          To set in 11-point type instead of 9-point.
% authoryear    To obtain author/year citation style instead of numeric.
\usepackage[utf8]{inputenc}          % UTF-8 Encoding
\usepackage[british]{babel}          % Use British English language settings
\usepackage{hyperref}                % Interactive PDF
\usepackage{xcolor}                  % Colours
\usepackage[inline]{enumitem}        % Inline enumerations
\usepackage{listings}                % Source code listings

\lstset{
 backgroundcolor=\color{white},   % choose the background color; you must add \usepackage{color} or \usepackage{xcolor}
 basicstyle=\ttfamily\footnotesize,        % the size of the fonts that are used for the code
 keywordstyle=\bfseries,
 commentstyle=\itshape,
 breakatwhitespace=true,         % sets if automatic breaks should only happen at whitespace
 breaklines=true,                 % sets automatic line breaking
 captionpos=b,                    % sets the caption-position to bottom
 escapeinside={\#*}{*\#},          % if you want to add LaTeX within your code
 extendedchars=true,              % lets you use non-ASCII characters; for 8-bits encodings only, does not work with UTF-8
 frame=none,	                   % adds a frame around the code
 keepspaces=true,                 % keeps spaces in text, useful for keeping indentation of code (possibly needs columns=flexible)
 numbers=none,                    % where to put the line-numbers; possible values are (none, left, right)
 rulecolor=\color{black},         % if not set, the frame-color may be changed on line-breaks within not-black text (e.g. comments (green here))
 showspaces=false,                % show spaces everywhere adding particular underscores; it overrides 'showstringspaces'
 showstringspaces=false,          % underline spaces within strings only
 showtabs=false,                  % show tabs within strings adding particular underscores
 tabsize=1,	                   % sets default tabsize to 2 spaces
 title=\lstname,                   % show the filename of files included with \lstinputlisting; also try caption instead of title
 caption={},
 belowcaptionskip=-1\baselineskip,
 xleftmargin=0.1\parindent,
 columns=fullflexible
}

% Define Links as a lst-language
\lstdefinelanguage{Links}{% 
  morekeywords={typename, fun, op, var, if, true, false, else, case, switch, handle, handler, shallowhandler, do, sig},%
  sensitive=t, % 
  keywordstyle=\color{red},
  emph={Comp,Bool,Int,Char,String,Choose,Return,Toss,Heads,Tails,Nothing,Just,Fail,Zero,Maybe},
  emphstyle={\color{blue}},
  comment=[l]{\#},% 
  escapeinside={(*}{*)},%
  morestring=[d]{"}%
}

\newcommand{\textapprox}{{\fontfamily{ptm}\selectfont\texttildelow}}
\newcommand{\wildarrow}{\linksify{\textapprox{}>}}
% Links style
\lstdefinestyle{links}{
  basicstyle=\linespread{1.0}\ttfamily\footnotesize,
  language=Links,
  literate= {~>}{{\wildarrow}}1
}

\lstset{style={links}}

%% TODOs and comments
\newcommand{\msgbox}[2]{{%
  \par\noindent\small\color{red}%
  \framebox{\parbox{\dimexpr\linewidth-2\fboxsep-2\fboxrule}{\textbf{#1:} #2}}%
}}
\newcommand{\todo}[1]{\msgbox{TODO}{#1}}

\newcommand{\sam}[1]{\msgbox{Sam}{#1}}
\newcommand{\dhil}[1]{\msgbox{Daniel}{#1}}
\newcommand{\kc}[1]{\msgbox{KC}{#1}}

\begin{document}
%% Remove SIGPLANCONF copyright space
\makeatletter
\def\@copyrightspace{\relax}
\makeatother

%% Set paper geometry
\special{papersize=8.5in,11in}
\setlength{\pdfpageheight}{\paperheight}
\setlength{\pdfpagewidth}{\paperwidth}

%% Obfuscate e-mail addresses
\newcommand{\camacuk}{@cam.ac.uk}
\newcommand{\edacuk}{@ed.ac.uk}
\newcommand{\contact}[2]{#1@#2}
\newcommand{\reachme}[1]{\hyperlink{mailto:\contact{#1}}{\contact{#1}}}

\titlebanner{DRAFT}    % These are ignored unless
\preprintfooter{} % 'preprint' option specified.

\title{Towards Compilation of Affine Algebraic Effect Handlers}
\subtitle{-- Extended Abstract --}
\authorinfo{Daniel Hillerström}
           {The University of Edinburgh}
           {~}%{\reachme{daniel.hillerstrom}}

\authorinfo{Sam Lindley}
           {The University of Edinburgh}
           {~}%{\reachme{sam.lindley}}

\authorinfo{KC Sivaramakrishnan}
           {The University of Cambridge}
           {~}

\maketitle

\begin{abstract}
  Algebraic effects and handlers provide a modular abstraction for
  effectful programming that allow programmers to separate effect
  signatures from their implementation. We present a compiler for the
  experimental language Links with effect handlers.
\dhil{TODO: Mention the OCaml backend}
\end{abstract}

\section{Motivation}
Algebraic effects and effect handlers \cite{Plotkin2013} afford a
modular and structured interface for programming with delimited
continuations.
\dhil{TODO: Motivation regarding the efficiency of handlers.}

Previous work focuses mostly on the design and abstraction of
handlers, but recent work has begun exploring efficient
implementations of handlers. Some notable mentions are the OCaml
Multicore project \cite{Dolan2015}, \citet{Kiselyov2015} \emph{freer
  monad} implementation in Haskell, and \citet{Wu2015} fusion
optimisation for (deep) handlers.

\section{Affine algebraic effect handlers}
An algebraic effect is given by a signature of \emph{abstract
  operations}. For example \emph{nondeterminism} is an algebraic
effect that is given by a nondeterministic choice operation called
\lstinline$Choose$. In Links, we may use this operation to implement a
coin toss:
\begin{lstlisting}
sig toss : Comp({Choose:Bool |e}, Toss)
fun toss() { if (do Choose) Heads else Tails }
\end{lstlisting}
This declares an \emph{abstract computation} \lstinline$toss$, which
invokes an operation \lstinline$Choose$ using the \lstinline$do$
primitive.  The \lstinline$sig$ keyword begins a signature, which
reads: \lstinline$toss$ is a computation with effect signature
\lstinline${Choose:Bool |e}$ and return value \lstinline$Toss$, whose
constructors are \lstinline$Heads$ and \lstinline$Tails$.  Links
employs row typing to support extensible effect signature, thus
\lstinline$e$ is an effect variable, which can be instantiated with
additional operations.

An effect handler instantiates a subset of the operations of an
abstract computation. For example, the following handler interprets
\lstinline$Choose$ randomly:
\begin{lstlisting}
sig randomResult : (Comp({Choose:Bool |e}, a)) -> 
                    Comp({Choose{_}   |e}, a)
handler randomResult {
  case Return(x) -> x
  case Choose(k) -> k(random() > 0.5)
}
\end{lstlisting}
The signature conveys that the handler interprets the operation
\lstinline$Choose$ and leaves potentially other operations
abstract. The notation \lstinline$Choose{_}$ denotes that the
operation is polymorphic in its presence.  The handler comprises two
clauses:
\begin{enumerate*}[label={\roman*)}]
\item the \lstinline$Return$-clause specifies how to handle the return
  value of the computation.
\item the parameter \lstinline$k$ in the \lstinline$Choose$-clause is
  the (delimited) continuation of the operation \lstinline$Choose$ in the
  computation.
\end{enumerate*}
We say that \lstinline$randomResult$ is a \emph{linear handler}, because
it invokes every continuation exactly once.

Alternatively, we may give an interpretation of \lstinline$Choose$
that enumerates every possible outcome by invoking the continuation
twice:
\begin{lstlisting}
sig allResults : (Comp({Choose:Bool |e},  a)) -> 
                  Comp({Choose{_}   |e}, [a])
handler allResults {
  case Return(x) -> [x]
  case Choose(k) -> k(true) ++ k(false)
}
\end{lstlisting}
Observe that the return value gets lifted into a singleton list. The
\lstinline$Choose$-clause concatenates the outcomes obtained by
interpreting the operation as \lstinline$true$ and \lstinline$false$,
respectively. We say that \lstinline$allResults$ is a \emph{multi-shot
  handler}.

Finally, we may have handlers that do not invoke continuations. These
handlers are familiarly known as \emph{exception handlers}. As an
example consider the following handler, which returns
\lstinline$Just$ the result of the computation or returns
\lstinline$Nothing$ if the operation \lstinline$Fail$ is performed:
\begin{lstlisting}
sig maybeResult : (Comp({Fail:Zero |e},       a)) -> 
                   Comp({Fail{_}   |e}, Maybe(a))
handler maybeResult {
  case Return(x) -> Just(x)
  case Fail(_)   -> Nothing
}
\end{lstlisting}
Here \lstinline$Zero$ is the empty type.  Linear and exception
handlers constitute \emph{affine handlers}.

\section{Compiler infrastructure}
\dhil{OCaml run-time maintains a stack of handlers; no optimisations.}

\section{Optimisations}
It is rather conservative to implement every handler as multi-shot. We
would like to recover the efficiency of affine handlers.

\paragraph{Linearisation} 
It is well-known that linear continuations have efficient
representations \cite{Bruggeman1996}.  We use Links' linear type
system to track the linearity of handlers. This enable us to
specialise the implementation of handlers during code generation.

\paragraph{Fusion} Handler fusion is a technique to prune the run-time
stack. Consider the composition
\lstinline$maybeResult(randomResult(toss))$. It gives
rise to a stack with two handlers.  We may use row types to guide when
handler fusion is sound. A sufficient criterion for fusion of two
handlers is that the intersection of their effect signatures is
empty. We fuse handlers clause by clause, thus for
\lstinline$maybeResult$ and \lstinline$randomResult$ we obtain:
\begin{lstlisting}
sig fused : (Comp({Choose:Bool,Fail:Zero |e}, a)) ->
             Comp({Choose{_}  ,Fail{_}   |e}, Maybe(a))
handler fused {
  case Return(x) -> var y = x; Just(y)
  case Fail(_)   -> Nothing
  case Choose(k) -> k(random() > 0.5)
}
\end{lstlisting}

\paragraph{Inlining} Operation inlining enables us to replace abstract
operations with their implementations, when these are statically
resolvable. Furthermore, the handler must affine and the operation
clause must invoke the continuation in tail position. For example, in
the application \lstinline$randomResult(toss)$ we may replace
\lstinline$do Choose$ with the computation \lstinline$random() > 0.5$ at compile-time.

\section{Acknowledgements}
The first author was supported by EPSRC CDT in Pervasive Parallelism
(grant EP/L01503X/1).  The second author was supported by EPSRC grant
number EP/K034413/1.  Thanks to OCaml Labs\dots

\bibliographystyle{abbrvnat} \softraggedright
\bibliography{references}

\end{document}