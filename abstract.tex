\documentclass[preprint,numbers]{sigplanconf}
% The following \documentclass options may be useful:

% preprint      Remove this option only once the paper is in final form.
% 10pt          To set in 10-point type instead of 9-point.
% 11pt          To set in 11-point type instead of 9-point.
% authoryear    To obtain author/year citation style instead of numeric.
\usepackage[utf8]{inputenc}          % UTF-8 Encoding
\usepackage[british]{babel}          % Use British English language settings
\usepackage{hyperref}                % Interactive PDF
\usepackage{xcolor}                  % Colours
\usepackage{listings}                % Source code listings

\lstset{
 backgroundcolor=\color{white},   % choose the background color; you must add \usepackage{color} or \usepackage{xcolor}
 basicstyle=\ttfamily\footnotesize,        % the size of the fonts that are used for the code
 keywordstyle=\bfseries,
 commentstyle=\itshape,
 breakatwhitespace=true,         % sets if automatic breaks should only happen at whitespace
 breaklines=true,                 % sets automatic line breaking
 captionpos=b,                    % sets the caption-position to bottom
 escapeinside={\#*}{*\#},          % if you want to add LaTeX within your code
 extendedchars=true,              % lets you use non-ASCII characters; for 8-bits encodings only, does not work with UTF-8
 frame=none,	                   % adds a frame around the code
 keepspaces=true,                 % keeps spaces in text, useful for keeping indentation of code (possibly needs columns=flexible)
 numbers=none,                    % where to put the line-numbers; possible values are (none, left, right)
 rulecolor=\color{black},         % if not set, the frame-color may be changed on line-breaks within not-black text (e.g. comments (green here))
 showspaces=false,                % show spaces everywhere adding particular underscores; it overrides 'showstringspaces'
 showstringspaces=false,          % underline spaces within strings only
 showtabs=false,                  % show tabs within strings adding particular underscores
 tabsize=1,	                   % sets default tabsize to 2 spaces
 title=\lstname,                   % show the filename of files included with \lstinputlisting; also try caption instead of title
 caption={},
 belowcaptionskip=-1\baselineskip,
 xleftmargin=0.1\parindent,
 columns=fullflexible
}

% Define Links as a lst-language
\lstdefinelanguage{Links}{% 
  morekeywords={typename, fun, op, var, if, true, false, else, case, switch, handle, handler, shallowhandler, do, sig},%
  sensitive=t, % 
  keywordstyle=\color{red},
  emph={Comp,Bool,Int,Char,String,Choose,Return,Toss},
  emphstyle={\color{blue}},
  comment=[l]{\#},% 
  escapeinside={(*}{*)},%
  morestring=[d]{"}%
}

\newcommand{\textapprox}{{\fontfamily{ptm}\selectfont\texttildelow}}
\newcommand{\wildarrow}{\linksify{\textapprox{}>}}
% Links style
\lstdefinestyle{links}{
  basicstyle=\linespread{1.0}\ttfamily\footnotesize,
  language=Links,
  literate= {~>}{{\wildarrow}}1
}

\lstset{style={links}}

%% TODOs and comments
\newcommand{\msgbox}[2]{{%
  \par\noindent\small\color{red}%
  \framebox{\parbox{\dimexpr\linewidth-2\fboxsep-2\fboxrule}{\textbf{#1:} #2}}%
}}
\newcommand{\todo}[1]{\msgbox{TODO}{#1}}

\newcommand{\sam}[1]{\msgbox{Sam}{#1}}
\newcommand{\dhil}[1]{\msgbox{Daniel}{#1}}
\newcommand{\kc}[1]{\msgbox{KC}{#1}}

\begin{document}
%% Remove SIGPLANCONF copyright space
\makeatletter
\def\@copyrightspace{\relax}
\makeatother

%% Set paper geometry
\special{papersize=8.5in,11in}
\setlength{\pdfpageheight}{\paperheight}
\setlength{\pdfpagewidth}{\paperwidth}

%% Obfuscate e-mail addresses
\newcommand{\camacuk}{@cam.ac.uk}
\newcommand{\edacuk}{@ed.ac.uk}
\newcommand{\contact}[2]{#1@#2}
\newcommand{\reachme}[1]{\hyperlink{mailto:\contact{#1}}{\contact{#1}}}

\titlebanner{}    % These are ignored unless
\preprintfooter{} % 'preprint' option specified.

\title{Towards Compilation of Affine Algebraic Effect Handlers}
\subtitle{-- Extended Abstract --}
\authorinfo{Daniel Hillerström}
           {The University of Edinburgh}
           {~}%{\reachme{daniel.hillerstrom}}

\authorinfo{Sam Lindley}
           {The University of Edinburgh}
           {~}%{\reachme{sam.lindley}}

\authorinfo{KC Sivaramakrishnan}
           {The University of Cambridge}
           {~}

\maketitle

\begin{abstract}
  Algebraic effects and handlers provide a modular abstraction for
  effectful programming that allow programmers to separate effect
  signatures from their implementation. We present a compiler for the
  experimental language Links with effect handlers.
\dhil{Mention the OCaml backend}
\end{abstract}

\section{Motivation}
Algebraic effects and effect handlers \cite{Plotkin2013} afford a
modular and structured interface for programming with delimited
continuations. 

\section{Affine algebraic effect handlers}
An algebraic effect is given by a signature of \emph{abstract
  operations}. For example \emph{nondeterminism} is an algebraic
effect that is given by a nondeterministic choice operation called
\lstinline$Choose$. In Links, we may implement it as follows:
\begin{lstlisting}
sig choose : () {Choose:Bool|e}-> Bool
fun choose() { do Choose }
\end{lstlisting}


\section{Compiler infrastructure}

\section{Optimisations}

\subsection{Linearisation}

\subsection{Rediscovering classical optimisations}

\section{Acknowledgements}
The first author was supported by EPSRC CDT in Pervasive Parallelism (grant EP/L01503X/1).
The second author was supported by EPSRC grant number EP/K034413/1.
Thanks to OCaml Labs\dots

\nocite{*} %% FIXME: remove

\bibliographystyle{abbrvnat} \softraggedright
\bibliography{references}

\end{document}